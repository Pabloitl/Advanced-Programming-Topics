\documentclass[12pt]{article}

\usepackage[margin=3cm]{geometry}
\usepackage{graphicx}
\usepackage{minted}
\usepackage{pdfpages}

\author{Pablo Vargas Bermúdez}

\begin{document}

\pagestyle{empty}

\includepdf[pages=-]{Portada}

\section*{Descripción}
Lea en el libro: Cómo programar en Java de Deitel el capítulo 23.-
Subprocesamiento múltiple poniendo especial interés en el apartado:
23.4.2 Administración de subprocesos con el marco de trabajo Executor,
para que realices la siguiente prueba al codigo que se muestra en la
figura 23.6

\begin{figure}[ht!]
  \centering
  \includegraphics[width=\textwidth]{figures/descripcion.png}
  \caption{Código a probar}
\end{figure}


Agrega otro 3 hilos a los mostrados en el programa, luego cambia la
linea de código 18 modificando el tipo de ejecutor.


\begin{itemize}
\item Cambiar linea 8 por: ExecutorService excS =
  Executors.newSingleThreadExecutor(); probar el programa varias veces
  y observar el funcionamiento.
\item Cambiar linea 8 por: ExecutorService excS =
  Executors.newFixedThreadPool(2); probar el programa varias veces y
  observar el funcionamiento.
\item Cambiar linea 8 por la línea original.  probar el programa
  varias veces y observar el funcionamiento.
\end{itemize}

Contesta las siguientes preguntas:

\begin{enumerate}
\item ¿Qué cambios notas en la ejecución del programa?
\item cambia el parámetro en ...newFixedThreadPool(\#); a tu gusto y
  observa el funcionamiento, ¿que conclusión sacas del valor colocado
  como argumento?
\item ¿Que ocurre cuando colocamos uno en el inciso 2)?
\item ¿podemos poner un valor negativo en el inciso 2)?
\item mueve la línea 26 del programa original en la posición de la
  linea 23, ¿que ocurre?  ¿cuál será la razón?
\end{enumerate}

Envía un archivo PDF que contenga una hoja de presentación, la
descripción de la tarea, el código fuente de los ejemplos y las
capturas de pantalla necesarias donde muestres el correcto
funcionamiento, así como tus respuestas a la preguntas planteadas.

Agrega preguntas o conclusiones

\section*{Código}

\subsection*{Executor}
\inputminted{Java}{Executor.java}
\subsection*{ExecutorSingle}
\inputminted{Java}{ExecutorSingle.java}
\subsection*{ExecutorFixed}
\inputminted{Java}{ExecutorFixed.java}

\section*{Ejecución}

\subsection*{Executor}
\begin{center}

  \includegraphics[width=.7\textwidth]{figures/executor1.png}
  \center{Prueba 1}
\end{center}

\begin{center}

  \includegraphics[width=.7\textwidth]{figures/executor2.png}
  \center{Prueba 2}
\end{center}

\begin{center}

  \includegraphics[width=\textwidth]{figures/executor3.png}
  \center{Prueba 3: shutdown() en línea 26}
\end{center}

\subsection*{ExecutorSingle}
\begin{center}

  \includegraphics[width=.7\textwidth]{figures/executorSingle1.png}
  \center{Prueba 1}
\end{center}

\begin{center}

  \includegraphics[width=.7\textwidth]{figures/executorSingle2.png}
  \center{Prueba 2}
\end{center}

\begin{center}

  \includegraphics[width=.7\textwidth]{figures/executorSingle3.png}
  \center{Prueba 3}
\end{center}

\subsection*{ExecutorFixed}
\begin{center}

  \includegraphics[width=.7\textwidth]{figures/executorFixed1.png}
  \center{Prueba 1}
\end{center}

\begin{center}

  \includegraphics[width=.7\textwidth]{figures/executorFixed2.png}
  \center{Prueba 2}
\end{center}

\begin{center}

  \includegraphics[width=\textwidth]{figures/executorFixed3.png}
  \center{Prueba 3: Parámetro negativo}
\end{center}

\section*{Cuestionario}

\begin{enumerate}
\item ¿Qué cambios notas en la ejecución del programa?

  CachedThreadPool: Parece correr todas las tareas de manera
  simultanea o en paralelo.

  SingleThreadExecutor: Parece correr una tarea a la vez pero en el
  orden en el que se le fueron asignando.

  FixedThreadPool: Parece correr solo un determinado numero de
  subprocesos al mismo tiempo haciendo que los subprocesos que no
  hayan alcanzado a ser ejecutados esperen en una cola hasta que haya
  lugar para ellos.

\item cambia el parámetro en ...newFixedThreadPool(\#); a tu gusto y
  observa el funcionamiento, ¿que conclusión sacas del valor colocado
  como argumento?

  Hace alusión al numero de subprocesos que pueden correr en paralelo,
  funciona como un Queue en cuanto termina de ejecutarse un proceso
  puede ejecutar otro que este esperando.

\item ¿Que ocurre cuando colocamos uno en el inciso 2)?

  Actúa como si solo existiera un hilo, solo se corre un subproceso a
  la vez y en cuanto termina uno entra el siguiente hasta que todos
  los subprocesos han sido ejecutados.

\item ¿podemos poner un valor negativo en el inciso 2)?

  No se puede colocar un valor negativo, en caso de que se intente se
  arrojará una excepción (IllegalArgumentException).

\item mueve la línea 26 del programa original en la posición de la
  linea 23, ¿que ocurre?  ¿cuál será la razón?

  Una vez que se manda a llamar al método shutdown() se liberan todos
  los recursos que este utilizando la clase ExecutorService por lo que
  ya no puede aceptar más subprocesos para ejecutar aunque si va a
  esperar a que los subprocesos que ya están en ejecución terminen de
  ejecutarse. Al liberar los recursos utilizados se permite que el
  colector de basura limpie los hilos que se hayan utilizado y permite
  que el programa pueda terminar normalmente.

\end{enumerate}

\section*{Conclusión}

El servicio Ejecutor nos da mas flexibilidad al tratar con varios
hilos además de proveernos con más control en la ejecución de los
hilos en el transcurso de nuestra aplicación.

Esto se convierte en una alternativa mucho más viable que lo usado en
tareas anteriores donde se instanciaba un objeto de tipo Thread para
crear un hilo partiendo de un Runnable, ya que si se realiza esto en
realidad no se tiene un buen control de los hilos que para ciertas
aplicaciones puede ser necesario tener.

Además al no ser nosotros los que inicializamos los hilos sino la API
de Java podemos estar seguros de que se estarían creando de la forma
más correcta.

\end{document}
