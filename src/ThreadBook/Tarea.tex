\documentclass[12pt]{article}

\usepackage[margin=3cm]{geometry}
\usepackage{pdfpages}
\usepackage{graphicx}
\usepackage{minted}

\author{Pablo Vargas Bermúdez}

\begin{document}

\pagestyle{empty}

\includepdf[pages=-]{Portada}

\section*{Descripción}

Lea el capitulo \# 14 del libro "Thinking in java, de Bruce Eckel"
realiza los programas de ejemplo que se muestran hasta antes de
procesos Demonio. De la página 669 a la 681

Envía un archivo PDF que contenga una hoja de presentación, la
descripción de la tarea, el código fuente de los ejemplos y las
capturas de pantalla necesarias donde muestres el correcto
funcionamiento,

Agrega preguntas o conclusiones

\section*{Código}

\subsection*{Counter 1}
\inputminted{Java}{Counter1.java}
\subsection*{SimpleThread}
\inputminted{Java}{SimpleThread.java}
\subsection*{Counter 2}
\inputminted{Java}{Counter2.java}
\subsection*{Counter 3}
\inputminted{Java}{Counter3.java}
\subsection*{Counter 4}
\inputminted{Java}{Counter4.java}

\section*{Ejecución}

\subsection*{Counter 1}

\begin{center}
  \includegraphics[width=.5\textwidth]{figures/counter1-1.png}
  \center{Antes de entrar al ciclo}
\end{center}

\begin{center}
  \includegraphics[width=.5\textwidth]{figures/counter1-2.png}
  \center{Dentro del ciclo}
\end{center}

\subsection*{SimpleThread}

\begin{center}
  \includegraphics[width=.5\textwidth]{figures/simplethread.png}
  \center{Creacion de varios hilos}
\end{center}

\subsection*{Counter 2}

\begin{center}
  \includegraphics[width=.5\textwidth]{figures/counter1-1.png}
  \center{Antes de entrar al ciclo}
\end{center}

\begin{center}
  \includegraphics[width=.5\textwidth]{figures/counter1-2.png}
  \center{Dentros del ciclo}
\end{center}

\subsection*{Counter 3}

\begin{center}
  \includegraphics[width=.5\textwidth]{figures/counter3-1.png}
  \center{Antes de entrar al ciclo}
\end{center}

\begin{center}
  \includegraphics[width=.5\textwidth]{figures/counter3-2.png}
  \center{Dentros del ciclo}
\end{center}

\subsection*{Counter 4}

\begin{center}
  \includegraphics[width=.5\textwidth]{figures/counter4-1.png}
  \center{Antes de entrar al ciclo}
\end{center}

\begin{center}
  \includegraphics[width=.5\textwidth]{figures/counter4-2.png}
  \center{Dentros del ciclo}
\end{center}

\begin{center}
  \includegraphics[width=.5\textwidth]{figures/counter4-3.png}
  \center{Después de parar algunos botones}
\end{center}

\section*{Conclusiones}

Cada vez las computadoras tienden a tener mas núcleos dentro de los
procesadores con los que se utilizan por lo que utilizar mas de un
hilo en los programas que se desarrollan como lo es Java en este caso
permite que el potencial de estos procesadores se aproveche con mayor
eficiencia que si solo utilizaremos un hilo.

Pienso que los hilos además de ser beneficiosos para el rendimiento
de una pieza de software también es de gran utilidad al diseñar la
aplicación ya que se puede romper en varias pequeñas acciones las
cuales pueden no necesitar estar ejecutándose todo el tiempo, pueden
responder a un evento como lo hemos visto hasta ahora en los
componentes gráficos y para que esto que se ha mencionado sea posible,
se utilizan los hilos.

\end{document}
