\documentclass[12pt]{article}

\usepackage[margin=3cm]{geometry}
\usepackage{graphicx}
\usepackage{pdfpages}
\usepackage{minted}

\author{Pablo Vargas Bermúdez}

\begin{document}

\pagestyle{empty}

\includepdf[pages=-]{Portada}

\section*{Descripción}

Lea en el libro: Cómo programar en Java de Deitel el capítulo 23.-
Subprocesamiento múltiple poniendo especial interés en el apartado:
23.5 Sincronización de Subprocesos

teclea el código de la figura 23.9 y comprueba su funcionamiento
varias veces para que notes que se llegan a perder valores.

Modifica el método de la clase ArregloSimple

\begin{figure}[ht!]
  \centering
  \includegraphics[width=\textwidth]{figures/descripcion.png}
  \caption{Modificaciones}
\end{figure}


Comprueba que ahora no existe perdida de información.

\begin{itemize}
\item ¿Crees que es importante la Sincronización?
\item Da un ejemplo real donde crees que esto podría funcionar.
\item ¿Cuando no sería necesaria la sincronización de procesos?
\end{itemize}

Envía un archivo PDF que contenga una hoja de presentación, la
descripción de la tarea, el código fuente de los ejemplos y las
capturas de pantalla necesarias donde muestres el correcto
funcionamiento, así como tus respuestas a la preguntas planteadas.

Agrega preguntas o conclusiones

\section*{Código}

\subsection*{Arreglo Simple}
\inputminted{Java}{ArregloSimple.java}
\subsection*{Escritor Arreglo}
\inputminted{Java}{EscritorArreglo.java}
\subsection*{Prueba Arreglo Compartido}
\inputminted{Java}{PruebaArregloCompartido.java}

\section*{Ejecución}

\subsection*{Sin ``synchronized''}
\begin{figure}[ht!]
  \centering
  \includegraphics[width=.7\textwidth]{figures/sin1.png}
  \caption{Sin ``syncrhonized'' 1}
\end{figure}

\begin{figure}[ht!]
  \centering
  \includegraphics[width=.7\textwidth]{figures/sin2.png}
  \caption{Sin ``syncrhonized'' 2}
\end{figure}

\subsection*{Con ``synchronized''}
\begin{figure}[ht!]
  \centering
  \includegraphics[width=.7\textwidth]{figures/con1.png}
  \caption{Con ``syncrhonized'' 1}
\end{figure}

\begin{figure}[ht!]
  \centering
  \includegraphics[width=.7\textwidth]{figures/con2.png}
  \caption{Con ``syncrhonized'' 2}
\end{figure}

\section*{Cuestionario}

\begin{itemize}
\item ¿Crees que es importante la Sincronización?

  Cuando existen hilos o subprocesos concurrentes, es decir, que
  modifican o trabajan con la misma variable si es muy importante que
  estén sincronizados porque de no ser así los hilos modificarían tal
  variable sin saber que esta ya ah sido cambiada por otro hilo, lo
  que puede dar resultados inesperados como se pudo observar en esta
  tarea en la que al modificar un mismo arreglo se perdían valores.

\item Da un ejemplo real donde crees que esto podría funcionar.

  Una aplicación en la que este concepto puede ser útil es en la
  gestión de archivos de un servidos ya que los métodos de
  lectura/escritura deben de estar sincronizados para evitar errores
  de que dos o más usuarios intenten escribir al mismo archivo
  justamente al mismo tiempo, ya que si se permite que escriban al
  mismo tiempo se pueden perder los contenidos de ese archivo como se
  pudo observar en el arreglo de esta tarea.

\item ¿Cuando no sería necesaria la sincronización de procesos?

  Cuando los procesos no modifiquen recursos en común o que no sean
  concurrentes, sino paralelos, no afectaría la falta de
  sincronización entre los procesos como en los ejemplos de tareas
  anteriores.

\end{itemize}

\section*{Conclusión}

La sincronización de los subprocesos y hilos es muy importante cuando
el objetivo es que cada subproceso o hilo modifique la misma variable
ya que muchas veces por no presumir que en todas las aplicaciones en
las que puedo pensar, si se va a modificar un valor varias veces se
necesita sabes cual era el valor de la variable antes de que el hilo
lo modificara y al poder ser cambiado por dos o más hilos al mismo
tiempo este principio no se sostiene y nos empezamos a enfrentar a
resultados inesperados.

\end{document}
