\documentclass[12pt]{article}

\author{Pablo Vargas Bermúdez}

\usepackage{pdfpages}
\usepackage{graphicx}
\usepackage[margin=3cm]{geometry}
\usepackage{minted}

\begin{document}
\pagestyle{empty}
\includepdf[pages=-]{Portada2}

\section*{Planteamiento}
Al ejercicio de la tarea \# 7, agregue al final de la información de
los renglones, despues de leer toda la información del archivo,
obtener la sumatoria de las columnas que tienen datos de Contagio,
Decesos y Recuperados la sumatoria de la columna. Tenga cuidado al
separar las columnas, ya que como vimos en clase existen registros que
contienen comillas y deben ser tomados en cuenta como una unidad vea
este extracto:

New York,US,2020-03-22T22:13:32,15793,117,0,42.1657,-74.9481

,"Korea, South",2020-03-22T11:13:17,8897,104,2909,35.9078,127.7669

,Switzerland,2020-03-22T23:13:18,7245,98,131,46.8182,8.2275

Y este otro

,Cambodia,2020-01-31T08:15:53,1,0,0

"London, ON",Canada,2020-02-04T00:03:11,1,0,0

,Finland,2020-01-31T08:15:53,1,0,0

en las dos primeras columnas existe la posibilidad que tengan comillas
por lo tanto deberán ser tomadas en cuenta como una sola entidad. En
el último ejercicio que tuvimos Creamos una Lista que contenía los
Países diferente, es decir concentraba todas las regiones que están en
registros aparte, me parece que solo obtuvimos la lista de países
diferentes y nos falto calcular los totales de cada país.

Coloca otra tabla debajo de la anterior de manera que muestre los
totales de infecciones, decesos y recuperaciones por país, ya que esta
información es que debemos VACIAR a una base de datos como siguiente
actividad.

\section*{Código}
\subsection*{Clase Data Builder}
\inputminted{Java}{DataBuilder.java}

\subsection*{Clase Covid}
\inputminted{Java}{Covid.java}

\subsection*{Clase Prueba}
\inputminted{Java}{Prueba.java}

\section*{Ejecución}
\begin{figure}[ht]
  \centering
  \includegraphics[width=\textwidth]{figures/run4.png}
  \caption{Primera prueba}
\end{figure}
\begin{figure}[ht]
  \centering
  \includegraphics[width=\textwidth]{figures/run5.png}
  \caption{Segunda prueba}
\end{figure}
\begin{figure}[ht]
  \centering
  \includegraphics[width=\textwidth]{figures/run6.png}
  \caption{Tercera prueba}
\end{figure}


\end{document}
