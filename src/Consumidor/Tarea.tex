\documentclass[12pt]{article}

\usepackage[margin=3cm]{geometry}
\usepackage{pdfpages}
\usepackage{minted}
\usepackage{graphicx}
\usepackage{float}

\author{Pablo Vargas Bermúdez}

\begin{document}

\pagestyle{empty}

\includepdf[pages=-]{Portada}

\section*{Descripción}

Lea en el libro: Cómo programar en Java de Deitel el capítulo 23.-
Subprocesamiento múltiple poniendo especial interés en el apartado:
23.6 Relación productor/consumidor sin sincronización, de las figuras
23.11 a la 23.15 viene un ejemplo completo, teclea las clases
necesarias para probar el funcionamiento de éste ejercicio, ejecuta
varias veces el programa para que compruebes que es inadecuada esta
solución.

Crea un diagrama UML de las clases utilizadas mostrando las relaciones
de las interfaces usadas.

Conteste las siguientes preguntas:

\begin{itemize}
\item ¿puedo crear 2 más hilos Productores y un solo consumidor? (en
  el main copia la linea 23) ejecuta el programa y da una conclusión
\item ¿se pueden crear un productor y 2 o más consumidores? (copia la
  linea 24 de main) ejecuta el programa y da una conclusión
\item Ahora crea varios productores y varios consumidores. ejecuta el
  programa y da una conclusión
\end{itemize}


Envía un archivo PDF que contenga una hoja de presentación, la
descripción de la tarea, el diagrama UML obtenido, el código fuente de
los ejemplos y las capturas de pantalla necesarias donde muestres el
correcto funcionamiento, así como tus respuestas a la preguntas
planteadas.

Agrega preguntas o conclusiones

\section*{Diagrama UML}

\begin{figure}[H]
  \centering
  \includegraphics[width=\textwidth]{figures/UML.png}
  \caption{Diagrama UML}
\end{figure}

\begin{center}
\end{center}

\section*{Código}

\subsection*{Bufer}
\inputminted{Java}{Bufer.java}
\subsection*{Consumidor}
\inputminted{Java}{Consumidor.java}
\subsection*{Productor}
\inputminted{Java}{Productor.java}
\subsection*{BuferSinSincronizacion}
\inputminted{Java}{BuferSinSincronizacion.java}
\subsection*{PruebaBuferCompartido}
\inputminted{Java}{PruebaBuferCompartido.java}

\section*{Ejecución}

\begin{figure}[H]
  \centering
  \includegraphics[width=.7\textwidth]{figures/normal.png}
  \caption{Ejecución con código original}
\end{figure}
\begin{figure}[H]
  \centering
  \includegraphics[width=.7\textwidth]{figures/productores.png}
  \caption{Más productores}
\end{figure}
\begin{figure}[H]
  \centering
  \includegraphics[width=.7\textwidth]{figures/consumidores.png}
  \caption{Más consumidores}
\end{figure}
\begin{figure}[H]
  \centering
  \includegraphics[width=.7\textwidth]{figures/varios.png}
  \caption{Varios productores y consumidores}
\end{figure}

\section*{Cuestionario}

\begin{itemize}
\item ¿puedo crear 2 más hilos Productores y un solo consumidor? (en
  el main copia la linea 23) ejecuta el programa y da una conclusión

  Si se pueden crear más de un Productor, lo que parece suceder es que
  los productores escriben mucho más rápido que lo que un solo
  consumidor puede contar por lo que la diferencia entre la cantidad
  contada y la producida aumenta, se tiende a contar menos de lo
  producido.

\item ¿se pueden crear un productor y 2 o más consumidores? (copia la
  linea 24 de main) ejecuta el programa y da una conclusión

  Si se pueden crear más de un Productor, lo que sucede en este caso
  es muy similar al anterior pero de cierta forma contrario, ahora se
  leen más valores de los que se producen por lo que la cuenta que los
  consumidores realizan termina siendo (en la mayoría de los casos)
  mayor a lo generado por el productor.

\item Ahora crea varios productores y varios consumidores. ejecuta el
  programa y da una conclusión

  En este caso, con un número igual de productores y consumidores
  mayor a uno se puede observar como la diferencia entre los números
  generados y los contados disminuye pero igual sigue siendo
  incorrecta, este caso tiende a ser bastante similar al que se podía
  observar con el código original con la diferencia en la cantidad de
  números generados y leídos producidos.

\end{itemize}

\section*{Conclusiones}

En esta tarea se puede apreciar la importancia de la sincronización
entre dos o más subprocesos o hilos que tienen recursos compartidos (a
los que se puede leer o escribir), precisamente por la falta de
sincronización se producen muchos errores a la hora de contar los
números producidos además de poder apreciar fenómenos como que un
productor escriba dos valores consecutivamente (sobre escribiendo el
valor de la variable) evitando que se cuenten los números que ah
producido o en el caso contrario se puede observar como en algunas
ocasiones se leen dos veces seguidas un mismo valor contándose dos
veces, por lo tanto afectando el resultado final y alejándolo del
valor correcto.

\end{document}
