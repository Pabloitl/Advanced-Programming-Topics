\documentclass[12pt]{article}

\usepackage[margin=3cm]{geometry}
\usepackage{graphicx}
\usepackage{pdfpages}
\usepackage{minted}
\usepackage{float}

\author{Pablo Vargas Bermúdez}

\begin{document}

\pagestyle{empty}

\includepdf[pages=-]{Portada}

\section*{Descripción}
Crea un programa que cree hilos para escribir tu nombre de la
siguiente manera:

Supongamos que nuestro nombre es "Alan Raúl López Estrada" y deseamos
que nuestro nombre se escriba 10 veces en la pantalla el Hilo tipo 1
escribe solamente los números,

El Hilo tipo 2 escribe solamente "Alan"
El Hilo tipo 3 escribe solamente "Raúl"
El Hilo tipo 4 escribe solamente "López"
El Hilo tipo 5 escribe solamente "Estrada"

Al correr nuestro programa deberemos obtener en consola lo siguiente:

\begin{figure}[H]
  \centering
  \includegraphics[width=.7\textwidth]{figures/descripcion.png}
  \caption{Muestra}
\end{figure}


En el main deberás de lanzar en total 20 hilos de los tipos indicados.

Modifica los tipos de hilo para que escriban tu nombre, quizás
necesites crear mas tipos de hilo.

Envía un archivo PDF que contenga una hoja de presentación, la
descripción de la tarea, el código fuente de los ejemplos y las
capturas de pantalla necesarias donde muestres el correcto
funcionamiento.

\section*{Código}

\subsection*{Nombre}
\inputminted{Java}{Nombre.java}
\subsection*{Prueba nombre}
\inputminted{Java}{PruebaNombre.java}

\section*{Ejecución}

\begin{figure}[H]
  \centering
  \includegraphics[width=.5\textwidth]{figures/run.png}
  \caption{Prueba de ejecución}
\end{figure}

\end{document}
